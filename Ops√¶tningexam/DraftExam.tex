\documentclass[12pt,a4paper]{article}
\usepackage{amsmath}
\usepackage{lmodern}
\usepackage[T1]{fontenc}
\usepackage[utf8]{inputenc}
\usepackage{setspace}
\onehalfspacing
\usepackage[danish]{babel}
\usepackage{graphicx}
\usepackage{subcaption}
\usepackage{pst-tree}
\usepackage{tikz}
\usetikzlibrary{trees}
\usepackage{amsfonts}
\usepackage{amsthm}
\usepackage[top=2.5cm, bottom=2.5cm, left=2.5cm, right=2.5cm]{geometry}
\usepackage{enumerate} 
\usepackage{amssymb}
\usepackage{csquotes}
\usepackage{fancyhdr}
\usepackage{times}
\usepackage{gauss}
\usepackage{fancyhdr}
\usepackage[bottom]{footmisc}
\pagestyle{fancy}
\usepackage{amsmath}
\usepackage{amssymb}
\usepackage{graphicx}
\usepackage{tikz}
\usepackage{lscape}
\newcommand*\circled[1]{\tikz[baseline=(char.base)]{
\node[shape=circle,draw,inner sep=2pt] (char) {#1};}}
\DeclareMathSizes{12}{12}{10}{10}
\interfootnotelinepenalty=10000
\usepackage{dsfont}
\usepackage{mathtools}
\usepackage{graphics}
\usepackage[danish]{babel} 
\usepackage{graphicx}
\usepackage{multirow, booktabs}
\usepackage[final]{pdfpages}
\usepackage{hyperref}

\lhead{}
\rhead{}
\chead{}

\begin{document}
\includepdf[pages=-]{forside.pdf}
\newpage
\onehalfspacing

\tableofcontents
\newpage
\section{Introduction}
This project predicts the valuation price of the danish housing market. By scraping \href{https://www.boliga.dk}{boliga.dk} we obtain a broad range of variables of potential interest, on all listed housing by august $22^{th}$, 2019. The AVI, \href{https://www.hvorlangterder.dk}{hvorlangterder.dk}, constructs measurements of distances from an address to conveniences and necessities such as hospitals, schools and shopping. Obtaining these distances for all of the listed housing and merging the two datasets, gives more opportunity for geographical-control. The data from hvorlangterder.dk will allow more micro-oriented differences within a given municipality. Which otherwise would not have been captured.  
\newline \underline{Måske en anden formulering ovenfor?} 
By applying supervised machine learning on our combined dataset.  
\section{Data description}
\subsection{Data construction}
Crime-rates. Traffic offences are not considered as a "serious" offence and are therefore neglected. 
\subsection{Exclusion of outliers}
A few of the listed housing, have been for sale for over a decade. These are considered outliers, which are overly priced considering their characteristics. The outliers are excluded, because they would otherwise only contribute to a bias of overestimation. Number of excluded housing.  
\newline
\begin{tabular}{ c | c | c | c | c | c | c | c} 
  \toprule
   & \multicolumn{2}{c|}{Dreng} & \multicolumn{2}{c}{Pige} \\ \hline
   & Hovedstaden & Sjælland & Syddanmark & Midtjylland & Nordjylland \\
   \midrule 
   Villa & 513,36 & 459,67 & 503,62 & 461,96 \\ \hline
   Rækkehus & 470,67 & 404,95 & 460,93 & 407,24 \\ \hline
   Ejerlejlighhed & 482,30 & 463,94 & 455,87 & 449,54 \\ \hline
   Fritidsbolig & 513,36 & 459,67 & 503,62 & 461,96 \\ \hline
   Kolonihave & 470,67 & 404,95 & 460,93 & 407,24 \\ \hline
   Andelsbolig & 482,30 & 463,94 & 455,87 & 449,54 \\ \hline
   Landejendom & 513,36 & 459,67 & 503,62 & 461,96 \\ \hline
   Helårsgrund & 470,67 & 404,95 & 460,93 & 407,24 \\ \hline
   Fritidsgrund & 482,30 & 463,94 & 455,87 & 449,54 \\ \hline
   Villalejlighed & 513,36 & 459,67 & 503,62 & 461,96 \\ \hline
   Lystejendom & 470,67 & 404,95 & 460,93 & 407,24 \\ 
    \bottomrule
\end{tabular}
\newline

\section{Methods}
\subsection{Supervised Machine Learning}
The objective by applying Machine Learning to these kinds of data, is to train a model that can predict pricing of housing listed in near future.
Two types of predictive models will be used for the sake of returning different outputs: Regression-model to return the pricing as a continuous output. As well as a classification-model for predicting the interval which the price is included in. 

\subsection{Regularization}
The use of a sizeable amount of regressors leads to a significant risk of over-fitting the prediction model. 
For the purpose of dealing with over-fitting, a form of regularization needs to be carried out. The Lasso-feature posses the attributes to do so. Regularization by Lasso, will penalize complexity of the model by adjustment of parameter estimates. This penalty will make the model less complex and more appropriate for prediction. \footnote{Big Data in Social Sciences, Forster et al. p. 173}\, \footnote{Python Machine Learning, Sebastian Raschka et al. p. 332}
\newline The regularization parameter $\lambda$ is determined by: $$Evt-optimeringsproblem$$

\subsection{Cross-Validation}
Selecting $\lambda$ 


\subsection{Variable selction}
Another convenient feature of Lasso, is the  


\section{Analysis}

\section{Results}

\section{Discussion}
\subsection{Data critique}
Another interesting prediction which could have been obtained using the same methods, could have been to estimate expected selling prices. This could have been done simply by scraping data on sold housing instead. This could be of more value for private agents, whose main interest should be the selling price of their housing. 
\newline 
In a prediction-model like this it is near impossible to evade some form of omitted variable bias. A significant amount of potential important factors can not be acquired. For an example the view from the listed housing will for sure have great impact of the valuation price. Another factor of interest could have been a evaluation of the condition of the housing, unfortunately the "statement of property" red. (tilstandrapport) are not publicly accessible.

\newpage
\section{Litterature}
Big Data in Social Sciences, Forster et al. 2017
\newline Python Machine Learning, Sebastian Raschka et al. 2017, $2^{nd}$editon
\end{document}